\documentclass[11pt,letterpaper]{article}

\usepackage{amsmath,amssymb,amsthm}
\usepackage{graphicx}
\usepackage{siunitx}
\usepackage{microtype}
\usepackage{tikz}
\usetikzlibrary{arrows,positioning,fit,decorations.pathmorphing}

% Set nice colors for TikZ pictures (also good grayscale balance)
\definecolor{OwlRed}{RGB}{ 255, 92, 168}
\definecolor{OwlGreen}{RGB}{ 90, 168, 0}
\definecolor{OwlBlue}{RGB}{ 0, 152, 233}
\definecolor{OwlYellow}{RGB}{ 242, 147, 24}
\colorlet{OwlViolet}{OwlRed!50!OwlBlue}
\colorlet{OwlBrown}{OwlRed!50!OwlGreen}
\colorlet{OwlOrange}{OwlRed!50!OwlYellow}
\colorlet{OwlCyan}{OwlGreen!50!OwlBlue}
\colorlet{red}{OwlRed}
\colorlet{green}{OwlGreen}
\colorlet{blue}{OwlBlue}
\colorlet{yellow}{OwlYellow}
\colorlet{violet}{OwlViolet}
\colorlet{brown}{OwlBrown}
\colorlet{orange}{OwlOrange}
\colorlet{cyan}{OwlCyan}

\title{Arbitrary Delay for \textsc{Fifo} P3F}
\author{Alex Striff}
\date{August 26, 2019}

\begin{document}
\maketitle

Since we can clock the read clock (\texttt{RCLK}) and write clock
(\texttt{WCLK}) for the \textsc{fifo} independently, we can make the delay be an
arbitrary function of state if
\[
  \textcolor{yellow}{f_\textsc{wclk}}
  = k\textcolor{blue}{\dot{\tau}} + \textcolor{red}{f_\textsc{rclk}}
\]
This may be achieved with the circuit of Fig.~\ref{fig:arb}.

\begin{figure}
  \centering
  \begin{tikzpicture}[%
      isosceles triangle apex angle=60,
      block/.style={rectangle,fill=#1,draw=#1!90,thick,rounded corners=4pt},
      block/.default={blue},
      signal/.style={midway,font=\footnotesize},
      conn/.style={thick,->},
      ampblock/.style={block=#1,isosceles triangle},
      ampblock/.default={blue},
      circblock/.style={block=#1,circle},
      circblock/.default={blue},
    ]
    \node[block=blue] (delay) at (0,0) {$\tau(y)$};
    \node[block=violet] (dt) at (1.5,0) {$\dv{t}$};
    \node[ampblock=violet] (amp) at (3,0) {$k$};
    \node[circblock=violet] (add) at (5,0) {$\Sigma$};
    \node[block=violet] (off) at (5,-1.5) {$V$};
    \node[block=violet] (vco) at (7,0) {VCO};
    \node[block=yellow] (wclk) at (9.5,0) {$f_\textsc{wclk}$};
    \node[block=red] (rclk) at (9.5,-1.5) {$f_\textsc{rclk}$};
    \draw[conn] (delay) -- (dt);
    \draw[conn] (dt) -- (amp);
    \draw[conn] (amp.east)+(-3pt,0) -- (add);
    \draw[conn] (rclk) -- (off);
    \draw[conn] (off) -- (add);
    \draw[conn] (add) -- (vco);
    \draw[conn] (vco) -- (wclk);
  \end{tikzpicture}
  \caption{Arbitrary delay circuit block diagram.}
  \label{fig:arb}
\end{figure}

I have already ordered some prototype parts that could be used to create this
circuit in hardware. They should be in the cardboard box of parts and stuff that
I left in the lab.

\begin{itemize}
  \item \texttt{TL082} \textsc{jfet}-input opamps (on breakout boards).
  \item A \textsc{vco} (voltage-controlled oscillator) part by \textsc{ti}.
  \item A \SI{4}{\MHz} crystal oscillator.
  \item A binary ripple-counter part (\textsc{cd}-series).
  \item Some \textsc{mux}es (maybe on breakout boards).
\end{itemize}

We just need to use opamps to differentiate $\tau(y)$ and then suitably offset
and scale that signal so that when $\dot{\tau} = \SI{0}{\s\per\s}$, the
input voltage to the \textsc{vco} will cause the \textsc{vco} to generate
a clock at $f_\textsc{rclk}$. Alternatively, make the differentiator inverting
and switch the role of \texttt{WCLK} and \textsc{RCLK}.

\end{document}



